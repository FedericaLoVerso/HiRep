\documentclass[a4paper]{article}

\usepackage{bm}
\usepackage{amsmath}
\usepackage{latexsym}


\newcommand{\tr}{\mathrm{tr}}
\newcommand{\gR}{\mathbf{R}}
\newcommand{\gr}{\mathbf{r}}
\newcommand{\point}{\; .}
\newcommand{\comma}{\; ,}

\newcommand{\be}{\begin{equation}}
\newcommand{\een}{\nonumber\end{equation}}
\newcommand{\ee}{\end{equation}}
\newcommand{\bea}{\begin{eqnarray}}
\newcommand{\eean}{\nonumber\end{eqnarray}}
\newcommand{\eea}{\end{eqnarray}}
\newcommand{\nonu}{\nonumber\\}
\newcommand{\sla}{\displaystyle{\not}}
\newcommand{\beq}{\begin{eqnarray}}
\newcommand{\eeq}{\end{eqnarray}}



\def\eq#1{Eq.~(\ref{#1})}
\def\eqs#1{Eqs.~(\ref{#1})}
\def\fig#1{Fig.~\ref{#1}}
\def\figs#1{Figs.~\ref{#1}}
\def\tab#1{Table~\ref{#1}}
\def\tabs#1{Tables~\ref{#1}}
\def\hf{\frac{1}{2}}
\def\ord#1{{\cal O}\left(#1\right)}
\def\la{\langle}
\def\ra{\rangle}
\def\und#1{\noindent\underline{#1}}
\def\lra{\longrightarrow}
\def\tr#1{{~\mathbf{tr}\left[#1\right]}}
\def\trD#1{{\mathbf{Tr}_D\left\{#1\right\}}}

                %% fields

                \def\psibar{\overline{\psi}}
                \def\chibar{\overline{\chi}}
                \def\rhobar{\bar{\rho}}
                \def\zetabar{\bar{\zeta}}


%\newcommand{\ind#1}{\textbf{\index#1}}
\def\red#1{{\textcolor{red}{#1}}}
\title{Estimation of disconnected contributions in HiRep}
\date{\today}

\begin{document}
\maketitle



\section{Conventions}
We choose the  hermitian basis of gamma matrices given in Tab.~\ref{tab:Gamma_basis}. Each element of the basis is refered by an index in $[0,15]$ shown in the table.

\begin{table}[ht]
\begin{center}
 \begin{tabular}[h]{cccccccc}
\hline
0 & 1 & 2 & 3 & 4 & 5 & 6 & 7 \\
$\gamma_5$ & $\gamma_1$  & $\gamma_2$  & $\gamma_3$ & $-i\gamma_0 \gamma_5$  & $-i\gamma_0 \gamma_1$  & $-i\gamma_0 \gamma_2$ &$-i\gamma_0 \gamma_3$   \\
\hline\hline
 8 & 9 & 10 & 11 & 12 & 13 & 14 & 15 \\
 $1$  & $-i\gamma_5 \gamma_1$  & $-i\gamma_5 \gamma_2$  & $-i\gamma_5 \gamma_3$  &  $\gamma_0$  & $-i\gamma_5\gamma_0\gamma_1 $  & $-i\gamma_5 \gamma_0\gamma_2$  & $-i\gamma_5 \gamma_0 \gamma_3$ \\
\hline
 \end{tabular}
\label{tab:Gamma_basis}
\caption{Basis used in HiRep}
\end{center}
\end{table}


\section{Singlet two-points functions}

Consider a gauge theory on a group $G$  coupled to $N_f$ fermions in
an arbitrary representation $R$.
Let us denote :
\beq
\label{eq:corr_scalar}
C(t,x_0) = \frac{1}{N_f} \sum_{\vec{x}}  \la \bar{q}  \Gamma  q (x) \bar{q} \Gamma q (x_0) \ra
\eeq

where $q,\bar{q}$ are the ($N_f$) quark fields and $\Gamma$ denotes an
arbitrary dirac structure. The $1/N_f$ factor is only there for convenience.
The Wick contractions read :
\beq
C(t,x_0) =  \sum_{\vec{x}} \la  -\tr{\Gamma S(x,x_0) \Gamma S(x_0,x)
}+ N_f  \tr{\Gamma S(x,x)} \tr{\Gamma  S(x_0,x_0)} \ra
\eeq

\section{stochastic evaluation of disconnected loops}
\subsection{General idea}

The simple one consist to evaluate stochastically the disconnected contribution without any variance reduction techniques. 
Considering a general volume source $\xi$, we define $\phi$ using the Dirac operator $D$:
\be
\phi = D^{-1} \xi
\ee
For a given element $X$ of the basis define in the previous section, we then have
 \begin{equation}
 \sum [\xi^* X \phi ]_R =
 \sum X M^{-1} + {\rm noise}
 \label{eq:B1}
 \end{equation}
 where the symbol $[\dots]_R$ refers to the average over $R$  samples of the stochastic source.

It should be observed that in evaluating the disconnected
contributions to the neutral  meson correlators each one of the two
quark loops arising from Wick contractions 
 must be averaged over completely {\em independent} samples of
stochastic sources for the purpose of avoiding unwanted biases. 



\subsection{Implemented source types}

We use XX noise sources. 
The user can switch between different source type, which have the
following correspondance :

\begin{itemize}
\item{ type $0$ : Pure volume source}
\item{ type $1$ : Gauge fixed wall source }
\item{ type $2$ : Volume sources with  time and spin dilution}
\item{ type $3$ : Volume sources with  time, spin and  color dilution}
\item{ type $4$ : Volume source with  time, spin, color and even-odd
    dilution} 
\item{type $6$ : Volume source with spin, color and even-odd dilution}
\end{itemize}


\section{Output}

The code does not perform any average on the stochastic noise or on the
dilution indices.  This allow to keep as much information as possible
and to vary the number of stochastic sources at the analysis level.


The indices are always :

\begin{verbatim}
#T #iGamma #iSrc #[color and/or e/o]  #Re #Im
\end{verbatim}

where iGamma refers to the index of the Gamma matrix defined in \tab{tab:Gamma_basis}
\section{Debugging options}


If the code is executed with the following additional args :
\begin{verbatim}
 -p <propagator name> -s <source_name>
\end{verbatim}
The code will read the two files and perform the contractions
accordingly, computing $\xi^\dagger\Gamma \psi$.

\end{document}


