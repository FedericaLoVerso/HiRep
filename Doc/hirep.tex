\documentclass{article}[12pt]

\usepackage{amsmath}
\usepackage{amsfonts}
\usepackage{amssymb}
\usepackage{graphicx}

\def\bfx{{\bf x}}

\bibliographystyle{JHEP}

\begin{document}


\section{Notation}
This section summarizes the main formulae that are used for
implementing the HMC for dynamical Wilson fermions in higher
representations. The Dirac operator is constructed following 
Ref.~\cite{Luscher:1996sc}, but using Hermitian generators 
%
\begin{equation}
T^{a\dagger}=T^a.
\end{equation}
%
For the fundamental representation, the normalization of the
generators is such that:
%
\begin{equation}
\mathrm{tr } \left(T^a T^b \right) = \frac12 \delta^{ab}.
\end{equation}
%
For a generic representation $R$, we define:
%
\begin{eqnarray}
\mathrm{tr }_R \left(T^a T^b \right) &=& T_R \delta^{ab}, \\
\sum_a \left(T^a T^a \right)_{AB} &=& C_2(R) \delta_{AB},
\end{eqnarray}
%
which implies: 
%
\begin{equation}
T_R = \frac{1}{N^2-1} C_2(R) d_R
\end{equation}
%
where $d_R$ is the dimension of the representation $R$.
The relevant group factors may be computed from the Young tableaux of the
representation of $SU(N)$ by using the formula:
%
\begin{equation}
C_2(R) =\frac{1}{2}\left(nN+ \sum_{i=1}^{m} n_i \left( n_i+1-2i
\right) - \frac{n^2}{N}\right)
\end{equation}
%
where $n$ is the number of boxes in the diagram, $i$ ranges over the
rows of the Young tableau, $m$ is the number of rows, and $n_i$ is the
number of boxes in the $i$-th row. 

\begin{table}[htp]
\begin{center}
\caption{Group invariants}
\label{table1}
\begin{tabular}{r|c|c|c}
R    & $d_R$               & $T_R$           & $C_2(R)$            \\
\hline
fund & $N$                 & $\frac12$       & $\frac{N^2-1}{2 N}$ \\        
Adj  & $N^2-1$             & $N$             & $N$ \\
2S   & $\frac{1}{2}N(N+1)$ & $\frac{N+2}{2}$ & $C_2(f) \frac{2(N+2)}{N+1}$ \\
2AS  & $\frac{1}{2}N(N-1)$ & $\frac{N-2}{2}$ & $C_2(f)
\frac{2(N-2)}{N-1}$ \\
\hline
\end{tabular}
\end{center}
\end{table}

\noindent
A generic element of the algebra is written as: $X=i X^a T^a$, and the
scalar product of two elements of the algebra is defined as:
%
\begin{equation}
(X,Y)= \mathrm{tr\ } \left(X^\dagger Y\right) = T_f X^a Y^a,
\end{equation}
%
\begin{equation}
\Vert X \Vert^2 = \mathrm{tr } \left(X^\dagger X\right)
 = \sum_{ij} \left| X_{ij} \right|^2
\end{equation}

\section{The Dirac operator}
The massless Dirac operator is written as in Ref.~\cite{Luscher:1996sc}:
%
\begin{equation}
D = \frac12 \left\{\gamma_\mu \left(\nabla_\mu + \nabla^*_\mu \right) 
- \nabla^*_\mu \nabla_\mu \right\}
\end{equation}
%
with
%
\begin{eqnarray}
\nabla_\mu\phi(x) &=& U^R (x,\mu)\phi(x+\mu) - \phi(x) \\
\nabla_\mu^*\phi(x) &=& \phi(x) - U^R (x-\mu,\mu)^\dagger\phi(x-\mu)
\end{eqnarray}
%
and therefore the action of the massive Dirac operator yields:
%
\begin{eqnarray}
D_m \phi(x) &=& (D+m) \phi(x) \nonumber \\
&=& - \frac12 \left\{ \left(1-\gamma_\mu\right) U^R(x,\mu) \phi(x+\mu) +
\left(1+\gamma_\mu\right) U^R(x-\mu,\mu)^\dagger \phi(x-\mu) - \right. 
\nonumber \\
& & \left. -(8+2m) \phi(x) \right\}, \label{DM}
\end{eqnarray}
%
where $U^R$ are the link variables in the representation $R$.

Rescaling the fermion fields by
$\sqrt{\kappa}=\left(\frac{2}{8+2m}\right)^{1/2}$, we can write the
fermionic action as:
%
\begin{equation}
S_f = \sum_{x,y} \phi^\dagger(x) D_m(x,y) \phi(y), 
\end{equation}
%
where
%
\begin{equation}
D_m(x,y) = \delta_{x,y} - \frac{\kappa}{2}
\left[(1-\gamma_\mu) U^R(x,\mu) \delta_{y,x+\mu} + 
(1+\gamma_\mu) U^R(x-\mu,\mu)^\dagger \delta_{y,x-\mu} \right],
\end{equation}
%
and the Hermitian Dirac operator is obtained as:
%
\begin{equation}
Q_m = \gamma_5 D_m. \label{QM}
\end{equation}
The fermionic determinant in the path integral can be represented by
introducing complex pseudofermionic fields:
\begin{equation}
\left(\det D_m\right)^{N_f} = 
\int \mathcal D \phi \mathcal D \phi^\dagger e^{-\phi^\dagger
  Q_m^{-N_f} \phi} \equiv
\int \mathcal D \phi \mathcal D \phi^\dagger e^{-S_\mathrm{pf}}.
\end{equation}

\section{Force for the HMC molecular dynamics}
The HMC Hamiltonian is given by:
\begin{equation}
\mathcal{H}=\mathcal{H}_\pi+\mathcal{H}_G+\mathcal{H}_F \, ,
\end{equation}
where
\begin{eqnarray}
\mathcal{H}_\pi &=& \frac{1}{2} \sum_{x,\mu} ( \pi(x,\mu) , \pi(x,\mu) ) = \frac{1}{2} T_f \sum_{a,x,\mu} \pi^a(x,\mu)^2 \, ,\\
\mathcal{H}_G &=& \beta \sum_{\mu<\nu} \left( 1- \frac{1}{N} \mathrm{Re\ tr\ } \mathcal{P}_{\mu\nu}\right) \, ,\\
\mathcal{H}_F &=& \phi^\dagger ( Q_m^2 - \beta )^{-l} \phi \, , \,\,\,\, l=\frac{N_f}{2}>0 \, ,
\end{eqnarray}
and we have introduced for each link variable a conjugate momentum in the algebra of the gauge group: $\pi(x,\mu)=i \pi^a(x,\mu) T_f^a$.
In the expression of $\mathcal{H}_F$ we omitted the sum over position, spin and color indices and we have also introduced an arbitrary shift $\beta$ for the matrix $Q_m^2$, as this will be useful in the discussion
for the RHMC algorithm.

The equation of motion for the link variables are given by (the $\dot{\square}$ indicates the derivative with respect to the molecular dynamics time):
\begin{equation}
\dot U(x\mu) = \pi(x,\mu) U(x,\mu)\, ,
\end{equation}
while the equation of motion for the momenta can be obtain as follows from the requirement that the hamiltonian
$\mathcal{H}$ is a conserved quantity:
\begin{equation}
0 = \dot{\mathcal{H}} = \dot{\mathcal{H}}_\pi + \dot{\mathcal{H}}_G + \dot{\mathcal{H}_F} \, . \label{HCONS}
\end{equation}

For the first two derivatives we have:
\begin{eqnarray}
\dot{\mathcal{H}}_\pi &=& \sum_{x,\mu} ( \pi(x,\mu) , \dot\pi(x,\mu) ) = T_f \sum_{x,\mu} \sum_a \pi^a(x,\mu) \dot\pi^a(x,\mu) \, \label{HDOTPI}\\
		\dot{\mathcal{H}}_G &=& \sum_{x,\mu} -\frac{\beta}{N} \mathrm{Re\ tr\ } \left[ \dot U(x,\mu) V^\dagger(x,\mu) \right] = \nonumber \\
	&=& \sum_{x,\mu} -\frac{\beta}{N} \mathrm{Re\ tr\ } \left[ \pi(x,\mu) U(x,\mu) V^\dagger(x,\mu) \right] \nonumber = \\
	&=& \sum_{x,\mu} \sum_a -\frac{\beta}{N} \pi^a(x,\mu) \mathrm{Re\ tr\ } \left[ i T^a_f U(x,\mu) V^\dagger(x,\mu) \right] \, , \label{HDOTG}
\end{eqnarray}
where $V(x,\mu)$ is the sum of the staples around the link $U(x,\mu)$.

The computation of the fermionic force goes as follows:
\begin{eqnarray}
\dot{\mathcal{H}}_F = -l\ \phi^\dagger (Q_m^2 - \beta)^{-(l+1)/2} \dot{(Q_m^2)} (Q_m^2 - \beta)^{-(l+1)/2} \phi \, . \label{FF1}
\end{eqnarray}
Defining:
\begin{eqnarray}
\eta &=& (Q_m^2 - \beta)^{-(l+1)/2} \phi \, , \\
\xi &=& Q_m \eta \, ,
\end{eqnarray}
and using the fact that the matrix $(Q_m^2-\beta)$ is hermitean, we can rewrite Eq.(\ref{FF1}) as
\begin{eqnarray}
\dot{\mathcal{H}}_F = - 2 l\ \xi^\dagger \dot{(Q_m)} \eta \, . \label{FF2}
\end{eqnarray}
Inserting the explicit form of $Q_m$, Eq.s~(\ref{QM}) and (\ref{DM}), into Eq.(\ref{FF2}) we obtain
\begin{multline}
\dot{\mathcal{H}}_F = l\ \mathrm{Re\ }\sum_{x,\mu} \xi(x)^\dagger \dot U^R(x,\mu) \gamma_5 (1-\gamma_\mu) \eta(x+\mu) \\ 
+ \xi(x+\mu)^\dagger \dot U^R(x,\mu)^\dagger \gamma_5 (1+\gamma_\mu) \eta(x) = \nonumber 
\end{multline}
\begin{multline}
\phantom{\dot{\mathcal{H}}_F} =  l\ \mathrm{Re\ }\sum_{x,\mu} \xi(x)^\dagger \dot U^R(x,\mu) \gamma_5 (1-\gamma_\mu) \eta(x+\mu) \\ 
+ \eta(x)^\dagger \dot U^R(x,\mu) \gamma_5 (1-\gamma_\mu) \xi(x+\mu)
\end{multline}
where the sum over spin and color indices is intended and we made explicit the fact the the whole 
expression is real.
We now use the fact that 
\begin{equation}
\dot U^R (x,\mu) = \pi^R(x,\mu) U^R(x,\mu) = i \pi^a(x,\mu) T^a_R U^R(x,\mu) \label{URDOT}
\end{equation}
Notice that, since we define $T^a_R(x,\mu) = R_* T^a(x,\mu)$, the $\pi^a(x,\mu)$ in the above 
equation are the same as those appearing in the expressions for $\dot{\mathcal{H}}_{\pi,G}$.
Using Eq.(\ref{URDOT}) in the expression for $\dot{\mathcal{H}}_{F}$ we find
(capitalized $Tr$ indicates the trace over color \textit{and} spin indices as opposed to the lower
 case $tr$ which is the trace over color only)
\begin{multline}
\dot{\mathcal{H}}_F = l\ \sum_{x,\mu} \sum_a \pi^a(x,\mu) \mathrm{Re\ Tr\ } \left[ iT^a_R U^R(x,\mu) \gamma_5 (1-\gamma_\mu) \right. \\
		\left. \left\{ \eta(x+\mu)\otimes\xi(x)^\dagger + \xi(x+\mu)\otimes\eta(x)^\dagger \right\} \right] \, . \label{HDOTF}
\end{multline}

Inserting Eq.s~(\ref{HDOTPI}),(\ref{HDOTG}),(\ref{HDOTF}) into Eq.~(\ref{HCONS}) we obtain the equation of motion
for the momenta $\pi^a(x,\mu)$
\begin{align}
\dot\pi^a(x,\mu) &= \dot\pi^a_G(x,\mu) + \dot\pi^a_F(x,\mu) \, , \label{PIDOT1}\\
\dot\pi^a_G(x,\mu) &= \frac{\beta}{N} \frac{1}{T_f} \mathrm{Re\ tr\ } \left[ i T^a_f U(x,\mu) V^\dagger(x,\mu) \right] \, ,\label{PIDOT2}\\
\dot\pi^a_F(x,\mu) &=- l \frac{1}{T_f} \mathrm{Re\ Tr\ } \left[ iT^a_R U^R(x,\mu) \gamma_5 (1-\gamma_\mu) \right. \nonumber\\
										&\quad\quad\quad	\left. \left\{ \eta(x+\mu)\otimes\xi(x)^\dagger + \xi(x+\mu)\otimes\eta(x)^\dagger \right\} \right]\, . \label{PIDOT3}
\end{align}

For sake of convenience we introduce the following projectors $P^a_R$ over the algebra in the rapresentation $R$:
\begin{equation}
P^a_R ( F ) = - \frac{1}{T_R} \mathrm{Re\ tr\ } \left[ i T^a_R F \right] \, ,
\end{equation}
we can be used to rewrite Eq.s~(\ref{PIDOT2})-(\ref{PIDOT3}) in a more compact form:
\begin{align}
\dot\pi^a_G(x,\mu) &= - \frac{\beta}{N} P^a_f \left( U(x,\mu) V^\dagger(x,\mu) \right) \, ,\\
\dot\pi^a_F(x,\mu) &= l \frac{T_R}{T_f} P^a_R \left( U^R(x,\mu) \mathrm{tr_{spin}} \left[ \gamma_5 (1-\gamma_\mu) \right. \right. \nonumber\\
										&\quad\quad\quad	\left. \left. \left\{ \eta(x+\mu)\otimes\xi(x)^\dagger + \xi(x+\mu)\otimes\eta(x)^\dagger \right\} \right] \right)\, . 
\end{align}



\section{HMC molecular dynamics}

The pseudofermionic action contributes to the force that determines
the time evolution of the conjugate momenta in the molecular
dynamics. Consider a variation of a link variable:
%
\begin{equation}
\delta U(x,\mu) = \omega(x,\mu) U(x,\mu),
\end{equation}
where $\omega$ is an element of the algebra $\mathcal G$; the
corresponding variation of the action defines the fermionic force,
$F(x,\mu)$:
%
\begin{equation}
\delta S_\mathrm{pf} = \left(\omega,F\right).
\end{equation}
%
The variation $\delta S_\mathrm{pf}$ can always be cast in the form:
%
\begin{equation}
\delta S_\mathrm{pf} = \sum_{x,y} \eta(x)^\dagger \delta Q_m(x,y)
\xi(y) + \xi(x)^\dagger \delta Q_m(x,y) \eta(y)
\end{equation}
%
where $\eta$ and $\xi$ are functions of $\phi$ that depend on the
number of flavours that we are considering. The expression needed for
the RHMC is considered in detail later. For fermions in a generic
representation $R$, and for an infinitesimal variation of the field
variable $U(z,\mu)$, we obtain:
%
\begin{align}
\delta Q_m(x,y) = &-\gamma_5 \left[(1-\gamma_\mu) \delta U^R(z,\mu)
\delta_{y,x+\mu} \delta_{x-\nu,z} \delta_{\mu\nu} \right. \nonumber \\ 
&~~~~~~~~~~~~~~~~~\left. + (1+\gamma_\mu) \delta U^R(z,\mu)^\dagger 
\delta_{y,x-\mu} \delta_{x-\nu,z} \delta_{\mu\nu} \right] \nonumber.
\end{align}
%
Using this expression for $\delta Q_m$, the variation of the fermionic
action is:
%
\begin{equation}
\delta S_\mathrm{pf} = \mathrm{Re} \left\{ \eta^\dagger(x) \gamma_5
\left(1-\gamma_\mu\right) \delta U^R(x,\mu) \xi(x+\mu) +
\eta^\dagger(x+\mu) \gamma_5
\left(1+\gamma_\mu\right) \delta U^R(x,\mu) \xi(x) \right\}
\end{equation}
%
In the following we denote by upper--case letters,
$A,B,\ldots=1,\ldots,d_R$, the colour indices in the representation
$R$, and by lower--case letters, $a,b,\ldots=1,\ldots,N$, the colour
indices in the fundamental representation.  The variation of the link
variable $U$ induces a variation of the field $U^R$:
%
\begin{eqnarray}
\delta U^R(x,\mu)^{AB} &=& {\mathcal C}_R(x,\mu)^{AB}_{ab} 
\delta U(x,\mu)_{ba}\\
\mathrm{or, }~~ \delta U^R(x,\mu) \hspace{15truept}&=& \mathrm{tr }_f 
\left[{\mathcal C}_R(x,\mu) \delta U(x,\mu) \right]\label{eq:dUR}.
\end{eqnarray}
%
The above equation defines the transition function ${\mathcal C}_R$,
which connects the variation of the field in the fundamental
representation to the variation of the field in the $R$
representation, and which needs to be computed for each representation
that we are interested in. In the second line, we have omitted the
open indices $A,B$, and the trace runs over the indices in the
fundamental representation, as suggested by the
subscript. The variation of the link variable in the $R$
representation can be rewritten as:
%
\begin{align}
\delta U^R &\equiv i \omega^a T^a_R \nonumber \\
&= i \omega^a \mathrm{tr}_f\left[T^a U \mathcal{C}_R \right] \label{VARUR}
\end{align}
%
and therefore:
%
\begin{equation}
\mathrm{tr}_f\left[T^a U \mathcal{C}_R \right] = T^a_R U^R.
\end{equation}
%
%% By introducing a new field:
%% %
%% \begin{equation}
%% {\mathcal U}^\prime(x,\mu)^{AB}_{ab} = U(x,\mu)_{ac} 
%% {\mathcal C}_R(x,\mu)^{AB}_{cb},
%% \end{equation}
%% the variations of the field $U^R$ can be rewritten as:
%% %
%% \begin{eqnarray}
%% \delta U^R(x,\mu) &=& -\left(\omega(x,\mu),{\mathcal U}^\prime(x,\mu)\right) \\
%% \delta U^R(x,\mu)^\dagger &=&
%% \left(\omega(x,\mu), {\mathcal U}^\prime(x,\mu)^\dagger
%% \right). 
%% \end{eqnarray}
%% %
%% The expressions above are particularly useful since they allow the
%% force to be readily written as:
%% %
%% \begin{eqnarray}
%% F(x,\mu)_{ab} &=&\eta(x)^\dagger \gamma_5 (1-\gamma_\mu)
%% {\mathcal U}^\prime(x,\mu)_{ab} \xi(x+\mu) - \nonumber \\
%% && - \eta(x+\mu)^\dagger \gamma_5 (1+\gamma_\mu)
%% {\mathcal U}^\prime(x,\mu)^\dagger_{ab} \xi(x).
%% \end{eqnarray}
%% %
%% where the open indices in the fundamental representation are written
%% explicitly, while the indices in the higher representation are
%% contracted and omitted in the equation. \\
Introducing a $d_R \times d_R$ matrix:  
%
\begin{equation}
P(x,y)^{AB} = \mathrm{tr}_\mathrm{Dirac}\, \left[
\gamma_5 \left(1-\gamma_\mu\right) \xi(x)^A \eta^\dagger(y)^B +
\gamma_5 \left(1-\gamma_\mu\right) \eta(x)^A \xi^\dagger(y)^B 
\right],
\end{equation}
%
where the $\mathrm{tr}_\mathrm{Dirac}$ indicates that Dirac indices
are contracted, the final expression for the force can be written
as a trace over colour indices in the $R$ representation:
%
\begin{equation}
\label{eq:force}
F^a(x,\mu) = -\frac{1}{T_f} \mathrm{Im}\, \mathrm{tr}_R \left[ T^a_R\, 
U^R(x,\mu) P(x+\mu,x) \right].
\end{equation}

\section{Checks of the MD force}
The formulae derived in the previous Section can be checked against two
known examples. The first, and almost trivial, check is obtained by
assuming that the representation $R$ is again the fundamental
representation. The well-known expression for the MD force for the
usual HMC is then recovered. 

The second case that has already been studied in the literature is the
case of fermions in the adjoint representationof the gauge group
SU($2$)~\cite{Donini:1996nr}. We agree with Eq.~(16) in
Ref.~~\cite{Donini:1996nr}, provided that we exchange the indices $a$
and $b$ in that formula.

\section{HMC Algorithm}

We briefly review the construction of the HMC algorithm \cite{??}.

Given the action $S(\phi)$ of a system of scalar fields $\phi$, our goal is to generate a Markov process with 
fixed probability distribution $P_S(\phi) = 1/Z \exp[-S(\phi) ]$. A sufficient condition to have such a 
Markov process is that it is ergodic and it satifies detailed balance:
\begin{equation}
P_S(\phi)P_M(\phi\rightarrow \phi') = P_S(\phi')P_M(\phi' \rightarrow \phi) \, .
\end{equation}
We define $P_M(\phi \rightarrow \phi')$ with the following three-step process:
\begin{enumerate}
\item we expand the configuration space with additional fields, the ``momenta" $\pi$ randomly chosen with probability
$P_k(\pi)$ such that $P_k(\pi)=P_k(-\pi)$ -- usually one takes $P_k(\pi)\propto \exp[-\pi^2/2]$; 
\item in the extended configuration space $(\phi, \pi)$, we generate a new configuration $(\phi',\pi')$ with probability
$P_h((\phi,\pi)\rightarrow(\phi',\pi'))$ such that 
\[P_h((\phi,\pi)\rightarrow(\phi',\pi')) = P_h((\phi',-\pi')\rightarrow(\phi,-\pi))\]
(reversibility condition);
\item we accept the new configuration $\phi'$ with probability 
\[P_A((\phi,\pi)\rightarrow(\phi',\pi')) = min \left\{ 1, \frac{P_S(\phi')P_k(\pi')}{P_S(\phi)P_k(\pi)} \right\} \, .\]
\end{enumerate}
It is easy to see that the resulting probability
\begin{equation}
P_M(\phi\rightarrow\phi') = \int d\pi d\pi' P_k(\pi) P_h((\phi,\pi)\rightarrow(\phi',\pi')) P_A((\phi,\pi)\rightarrow(\phi',\pi')) \, ,
\end{equation}
satisfies detailed balance. Care must be taken to ensure ergodicity.

As already stated, the distribution $P_k(\pi)$ is generally taken to be gaussian (this should also garantee ergodicity).
The process $P_h$ is instead identified with the hamiltonian flow of a yet unspecified Hamiltonian $H$ in the phase
space $(\phi,\pi)$ (giving to $\pi$ the meaning of ``momenta"). The time reversal symmetry of classical dynamics equation of motion
garantees the reversibility condition. The resulting probability $P_h$ is then a delta function (the process is completly deterministic).
Numerical integration to a given accuracy will result in a broader distribution and care must be taken to garantee the reversibility condition 
in this case.
Since we want a high acceptance rate (low correlation among the configurations), we must carefully choose the Hamiltonian $H$.
One simple way is to take $P_k$ to be gaussian and to define $H(\pi,\phi)=-\ln [P_k(\pi) P_S(\phi)] = \pi^2/2 + S(\phi)$ 
(omitting irrelevant constants). If $H$ is exactly conserved by the process $P_h$ then the acceptance probability is 1.

When fermionic degrees of freedom are present in the action $S$, we can first integrate them out, resulting in a non
local bosonic action and then apply the above scheme. In practice, to deal with a non-local action is not convienent
from a numerical point a view and stocastic extimates are used.

Consider a quadratic fermionic term  in the action: $S(\bar\psi,\psi) = \bar\psi M \psi$ with a generic interaction
matrix $M$ possibly function of the other bosonic fields $\phi$. The contribution of this term to the partition function
is $\int d\bar\psi d\psi \exp [ -S(\bar\psi,\psi)] = det[M]$. Assuming that the matrix $M$ is positive definite,
we can rewrite $det[M]=\int d\bar\eta d\eta \exp[ \bar\eta (M)^{-1} \eta ]$, where $\bar\eta$,$\eta$ are two new 
complex bosonic fields, called pseudofermions.
This term can be taken into account generating random pseudofermions $\bar\eta$, $\eta$ with the desidered probability
distribution and keeping then fixed during the above HMC configuration generation for the remaining bosonic fields $\phi$.



\section{RHMC formulation}

In order to discuss the RHMC implementation we follow the notation in
Ref.~\cite{Clark:2005sq}: starting from a fermionic action:
%
\begin{equation}
S_\mathrm{pf}=-\sum_{x,y} \phi^\dagger(x) \mathcal F(Q^2_m)_{x,y} \phi(y),
\end{equation}
%
where $\mathcal F$ is a generic function of $Q^2_m$, we construct two
rational approximations:
%
\begin{align}
r(Q^2_m) &= \sum_k \frac{\alpha_k}{Q^2_m + \beta_k} 
\nonumber \\
\bar r(Q^2_m) &= \sum_k \frac{\bar \alpha_k}{Q^2_m + \bar \beta_k}, \nonumber
\end{align}
%
such that:
%
\begin{equation}
r^2(x) \approx \bar r(x) \approx \mathcal F.
\end{equation}
%
The number of terms appearing in each approximation depends on the
degree of accuracy that is required. \\
The first approximation, $r$, is used to generate pseudofermionic fields at
the beginning of the trajectories using an heat--baht, while the
second approximation, $\bar r$ is used during the MD evolution. \\
The explicit expression for the action:
%
\begin{equation}
S_\mathrm{pf}= - \sum_k \bar \alpha_k \phi^\dagger \left[Q^2_m + \bar
  \beta_k \right]^{-1} \phi
\end{equation} 
%
yields the following variation:
%
\begin{equation}
\delta S_\mathrm{pf} = \sum_k \bar\alpha_k \phi_k^\dagger
\left\{Q_m \delta Q_m + \delta Q_m Q_m \right\} \phi_k,
\end{equation}
%
where $\phi_k=\left[Q^2_m + \bar
  \beta_k \right]^{-1} \phi$.

\section{Conventions used in the code}

\subsection{Representations}

The hermitean generators $T^a_f$ for the fundamental representation used are of the form:
\begin{equation}
\begin{pmatrix} 
0&1&0&\dots\\
1&0&0&\dots\\
0&0&0&\dots\\
\hdotsfor{4}
\end{pmatrix}\, ,
\begin{pmatrix} 
0&i&0&\dots\\
-i&0&0&\dots\\
0&0&0&\dots\\
\hdotsfor{4}
\end{pmatrix}\, ,
\begin{pmatrix} 
1&0&0&\dots\\
0&1&0&\dots\\
0&0&-2&\dots\\
\hdotsfor{4}
\end{pmatrix}\, ,
\end{equation}
normalized so that $T_f=1/2$. The generators in for the other representations 
will be obtained in the following.

We first give the explicit form for the representation functions $R$ which map 
$U\rightarrow U^R$. We define for each representation an orthonormal base $e_R$ for 
the appropriate vector space of matrices. 

For the Adjoint representation we define the base $e_{Adj}$ for the $N\times N$ 
traceless hemitean matrices to be $e_{Adj}^a=T^a_f/\sqrt{T_f}$, $a=1,\dots,N^2-1$ 
(i.e. proportional to the generators of the fundamental representation and 
normalized to 1.)

For the two-index Symmetric representation the base $e^{(ij)}_{S}$, with $i\le j$, for 
the $N\times N$ symmetric matrices is given by:
\begin{eqnarray}
i\neq j \, ,\,\,\,&e^{(ij)}_S&=\frac{1}{\sqrt{2}}\begin{pmatrix} 
0&1&0&\dots\\
1&0&0&\dots\\
0&0&0&\dots\\
\hdotsfor{4}
\end{pmatrix}\, , \\
i=j \, ,\,\,\,&e^{(ii)}_S&=\begin{pmatrix} 
0&0&0&\dots\\
0&1&0&\dots\\
0&0&0&\dots\\
\hdotsfor{4}
\end{pmatrix}\, , 
\end{eqnarray}
where the non zero entries are at position $(i,j)$, etc.

For the two-index Antisymmetric representation the base $e^{(ij)}_{AS}$, with $i<j$, for 
the $N\times N$ symmetric matrices is given by:
\begin{equation}
e^{(ij)}_{AS}=\frac{1}{\sqrt{2}}\begin{pmatrix} 
0&1&0&\dots\\
-1&0&0&\dots\\
0&0&0&\dots\\
\hdotsfor{4}
\end{pmatrix}\, , 
\end{equation}
where, as above, the non zero entries are at position $(i,j)$.

The maps $R$ are explicitly given by:
\begin{eqnarray}
(R^{Adj} U)_{ab} &=& U^{Adj}_{ab} = \mathrm{tr\ }\left[ e^a_{Adj} U e^b_{Adj} U^\dagger\right]\,\, , a,b=1,\dots,N^2-1\, ,\\ 
(R^{S} U)_{(ij)(lk)} &=& U^{S}_{(ij)(lk)} = \mathrm{tr\ }\left[ (e^{(ij)}_{S})^\dagger U e^{(lk)}_S U^T\right]\,\, , i\le j, l\le k\, ,\\ 
(R^{A} U)_{(ij)(lk)} &=& U^{A}_{(ij)(lk)} = \mathrm{tr\ }\left[ (e^{(ij)}_{A})^\dagger U e^{(lk)}_A U^T\right]\,\, , i< j, l< k\, .
\end{eqnarray}

The generators $T_R^a$ used are defined as the image of the generators in the fundamental
under the differential of the maps $R$ defined above: $T^a_R = R_* T^a_f$.
Explicit expression can easily be worked out form the definition above.
The invariants $T_R$ and $C_2(R)$ for the generators defined in this way are given in 
Table~(\ref{table1}).

\subsection{$\gamma$ matrices}
We use the chiral representation for the Dirac $\gamma$ matrices:
\begin{equation}
\gamma_\mu=
\begin{pmatrix}
0&e_\mu\\
e_\mu^\dagger&0
\end{pmatrix}\, ,
\end{equation}
where $e_\mu$ are $2\times 2$ matrices given by: $e_0=-1$, $e_k=-i\sigma_k$,
\begin{equation}
\sigma_1=
\begin{pmatrix}
0&1\\
1&0
\end{pmatrix},\,\,
\sigma_2=
\begin{pmatrix}
0&-i\\
i&0
\end{pmatrix},\,\,
\sigma_3=
\begin{pmatrix}
1&0\\
0&-1
\end{pmatrix}\, .
\end{equation}
We have:
\begin{equation}
\gamma_5=\gamma_0\gamma_1\gamma_2\gamma_3=
\begin{pmatrix}
1&0\\
0&-1
\end{pmatrix}\, .
\end{equation}



\section{Two--point functions}
This is a summary of the formulae used for the mesonic two--pt
functions. 

\bigskip

\noindent
Let $\Gamma$ and $\Gamma^\prime$ be two generic matrices in the
Clifford algebra, we define the two--pt function:
%
\begin{equation}
f_{\Gamma\Gamma^\prime}(t) = \sum_\bfx \langle \bar\psi(\bfx,t) \Gamma
\psi(\bfx,t) \bar\psi(0) \Gamma^\prime \psi(0) \rangle
\end{equation}
%
Performing the Wick contractions yields:
%
\begin{eqnarray}
\langle \bar\psi(\bfx,t) \Gamma
\psi(\bfx,t) \bar\psi(0) \Gamma^\prime \psi(0) \rangle &=&
- \mathrm{tr} \left[ \Gamma S(x-y) \Gamma^\prime S(y-x) \right] 
\nonumber \\
&=& - \mathrm{tr} \left[ \Gamma S(x-y) \Gamma^\prime \gamma_5 
S^\dagger(x-y) \gamma_5 \right] \nonumber 
\end{eqnarray}
%
In practice we invert the Hermitean Dirac operator $\gamma_5 D$ by
solving the equation:
%
\begin{equation}
Q_{AB}(x-y) \eta^{\bar A,x_0}_B(y) = \delta_{A,\bar A} \delta_{x,x_0}
\end{equation}
%
where $A=\{a,\alpha\}$ is a collective index for colour and spin, and
$\bar A$, $x_0$ are the position of the source for the inverter. 

Using the field $\eta$ that we obtain from the inverter, the
correlator above becomes:
%
\begin{equation}
\langle \ldots \rangle = - \tilde \Gamma_{AB} \eta^{C,y}_B(x)
\tilde \Gamma^\prime_{CD} \eta^{D,y}_A(x)^*
\end{equation}
where $\tilde \Gamma= \gamma_5 \Gamma$, and $\tilde \Gamma^\prime =
\gamma_5 \Gamma^\prime$.

\bibliography{hirep}

\end{document}
